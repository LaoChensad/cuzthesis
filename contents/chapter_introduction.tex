\chapter{绪论}\label{chap:introduction}

\section{课题背景}

考虑到许多同学可能缺乏\LaTeX{}使用经验,cuzthesis在参考了ucasthesis模板的基础
上,将\LaTeX{}的复杂性高度封装,开放出简单的接口,以便轻易使用。同时,对用
\LaTeX{}撰写论文的一些主要难题,如制图、制表、文献索引等,进行了详细说明,并提供
了相应的代码样本,理解了上述问题后,对于初学者而言,使用此模板撰写学位论文将不存
在实质性的困难。所以,如果你是初学者,请不要直接放弃,因为同样为初学者的我,十分
明白让\LaTeX{}简单易用的重要性,而这正是cuzthesis与ucasthesis所追求和体现的。

此浙传毕业论文模板cuzthesis基于国科大莫晃锐制作的ucasthesis模板发展而来。当前
cuzthesis模板满足最新的浙江传媒学院本科毕业论文撰写要求和封面设定。兼顾操作系
统:Windows,Linux,macOS 和\LaTeX{}编译引擎:\hologo{pdfLaTeX}、
\hologo{XeLaTeX}、\hologo{LuaLaTeX}。支持中文书签、中文渲染、中文粗体显示、拷贝
PDF中的文本到其他文本编辑器等特性。此外,对模板的文档结构进行了精心设计,撰写了
编译脚本提高模板的易用性和使用效率。

cuzthesis的目标在于简化毕业论文的撰写,利用\LaTeX{}格式与内容分离的特征,模板将
格式设计好后,作者可只需关注论文内容。同时,cuzthesis有着整洁一致的代码结构和扼
要的注解,对文档的仔细阅读可为初学者提供一个学习\LaTeX{}的窗口。此外,模板的架构
十分注重通用性,事实上,与ucasthesis一样,cuzthesis不仅是浙传毕业论文模板,同
时,通过少量修改即可成为使用\LaTeX{}撰写中英文文章或书籍的通用模板,并为使用者的
个性化设定提供了接口。

\section{系统要求}\label{sec:system}

\href{https://github.com/mohuangrui/ucasthesis}{\texttt{ucasthesis}} 宏包可以在
目前主流的 \href{https://en.wikibooks.org/wiki/LaTeX/Introduction}{\LaTeX{}} 编
译系统中使用,例如C\TeX{}套装 (请勿混淆C\TeX{}套装与ctex宏包。C\TeX{}套装是集成
了许多\LaTeX{}组件的\LaTeX{}编译系统,因已停止维护,\textbf{不再建议使用}。
\href{https://ctan.org/pkg/ctex?lang=en}{ctex} 宏包如同cuzthesis,是\LaTeX{}命令
集,其维护状态活跃,并被主流的\LaTeX{}编译系统默认集成,是几乎所有\LaTeX{}中文文
档的核心架构。)、MiK\TeX{}(维护较不稳定,\textbf{不太推荐使用})、\TeX{}Live。
而文本编辑器方面则包括:Visual Studio Code(简称vscode,强烈推荐)、
\hologo{TeX}、Emacs、Vim等。推荐的
\href{https://en.wikibooks.org/wiki/LaTeX/Installation}{\LaTeX{}编译系统}和
\href{https://en.wikibooks.org/wiki/LaTeX/Installation}{\LaTeX{}文本编辑器}为
\begin{center}
    %\footnotesize% fontsize
    %\setlength{\tabcolsep}{4pt}% column separation
    %\renewcommand{\arraystretch}{1.5}% row space 
    \begin{tabular}{lcc}
        \hline
        %\multicolumn{num_of_cols_to_merge}{alignment}{contents} \\
        %\cline{i-j}% partial hline from column i to column j
        操作系统 & \LaTeX{}编译系统 & \LaTeX{}文本编辑器\\
        \hline
        Linux &
        \href{https://www.tug.org/texlive/acquire-netinstall.html}{\TeX{}Live
        Full} & \href{https://code.visualstudio.com/download}{vscode}、
        \href{https://www.texstudio.org/}{TeXstudio}、Emacs、Vim\\
        MacOS & \href{https://www.tug.org/mactex/}{Mac\TeX{} Full} &
        \href{https://code.visualstudio.com/download}{vscode}、
        \href{https://www.texstudio.org/}{TeXstudio}、Emacs\\
        Windows &
        \href{https://www.tug.org/texlive/acquire-netinstall.html}{\TeX{}Live
        Full} & \href{https://code.visualstudio.com/download}{vscode}、
        \href{https://www.texstudio.org/}{TeXstudio}\\
        \hline
    \end{tabular}
\end{center}

\LaTeX{}编译系统,如\TeX{}Live(Mac\TeX{}为针对macOS的\TeX{}Live),用于提供编译
环境;\LaTeX{}文本编辑器 (如vscode) 用于编辑\TeX{}源文件。请从各软件官网下载安装
程序,勿使用不明程序源。\textbf{\LaTeX{}编译系统和\LaTeX{}编辑器分别安装成功后,
即完成了\LaTeX{}的系统配置},无需其他手动干预和配置。若系统原带有旧版的\LaTeX{}
编译系统并想安装新版,请\textbf{先卸载干净旧版再安装新版}。

\section{问题反馈}

关于\LaTeX{}的知识性问题,请查阅
\href{https://github.com/mohuangrui/ucasthesis/wiki}{ucasthesis和\LaTeX{}知识小
站} 和 \href{https://en.wikibooks.org/wiki/LaTeX}{\LaTeX{} Wikibook}。

关于模板编译和样式设计的问题,请\textbf{先仔细阅读此说明文档,特别是“常见问题”
(章节~\ref{sec:qa})}。若问题仍无法得到解决,请\textbf{先将问题理解清楚并描述清
楚,再将问题反馈}至
\href{https://github.com/xiehao/CUZThesis/issues}{Github/ucasthesis/issues}。

欢迎大家有效地反馈模板不足之处,一起不断改进模板。希望大家向同事积极推广
\LaTeX{},一起更高效地做科研。

\section{模板下载}

\begin{center}
    \href{https://github.com/xiehao/CUZThesis}{Github/cuzthesis}: \url{https://github.com/xiehao/CUZThesis}
\end{center}
